% Metódy inžinierskej práce

\documentclass[10pt,twoside,slovak,a4paper]{article}

\usepackage[slovak, russian]{babel}
%\usepackage[T1]{fontenc}
\usepackage[IL2]{fontenc} % lepšia sadzba písmena Ľ než v T1
\usepackage[utf8]{inputenc}
\usepackage{graphicx}
\usepackage{url} % príkaz \url na formátovanie URL
\usepackage{hyperref} % odkazy v texte budú aktívne (pri niektorých triedach dokumentov spôsobuje posun textu)

\usepackage{cite}
%\usepackage{times}

\pagestyle{headings}

\title{Web modelovanie: prečo je to potrebné\thanks{Semestrálny projekt v predmete Metódy inžinierskej práce, ak. rok 2021/22, vedenie: Arman Vardanyan}} % meno a priezvisko vyučujúceho na cvičeniach

\author{Arman Vardanyan\\[2pt]
	{\small Slovenská technická univerzita v Bratislave}\\
	{\small Fakulta informatiky a informačných technológií}\\
	{\small \texttt{xvardanyan@stuba.sk}}
	}

\date{\small 09.10.2021} % upravte

\begin{document}

\maketitle

\begin{center}
    
\end{center}

\section{Úvod}

Modelovanie je jedna z hlavných fáz vytvárania webovej stránky. Počas neho odpovedáme na otázky podľa typu: “prečo vytvaráme tu webovu stránku”, “ aké sú naše ciele ”, “ ako implementovať naše nápady ”, a nakoniec “ ako bude náš web vyzerať a fungovať ”. V tomto článku sa pozrieme na to, ako projektovať webovú stránku, hlavné fázy návrhu webových stránok, aké nástroje a programy sa používajú počas práce, prečo nevhodne ignorovať modelovanie web stranky, aj keď projekt je veľmi malý. Prečo minuv čas na projektovanie my ušetrime viacej času a peniaze na nasledujúcich fázach projekta, a aké psychologické metódy používajú profesionáli aby zaujať pozornosť zakaznika. A nakoniec spoločne pochopíme, prečo web modelovanie je potrebne.


\section{Čo je to web modelovanie}

dskljfaskjdfaklsdfaksdfksafdh

\section{ako Projektovať webovú stránku}

sadfasdfsdfaasdfasdf

\section{fázy navrhu webových stránok}

sadfljadfasdklfaskdjlfasdkjlfh

\section{programy}

один из самых сложных вопросов, это какие программы выбрать для веб моделирование. Каждый начинающий UI/UX специалист сталкивается с такой проблемой, что во первых программ очень много, как платных, так и бесплатных, и во вторых нету одной программы в которой можно сделать все задачи, каждая программа хорошо справляется только с несколькими задачами. В этой главе мы рассмотрим какие программы существуют, и с какими задачами можно справится с помощю этих программ.

1. Adobe Photoshop

несмотря на то что Adobe Photoshop не полностью соответсвует для веб моделирования, это самая популярная программа для дизайнеров. Уже очень долгое время Adobe Photoshop используется для создания интерфейсов. У В нем огромное количество возможностей, но веб-дизайнеры используют его для рисования интерфейсов, создания растровой графики, обработки фотографий. Графические дизайнеры примерно для таких же целей.Так что если вы фрилансер и вы не знаете кто будет верстать ваш макет, фотошоп это по-прежнему универсальный и лучший вариант.

2. Adobe Illustrator

Adobe Illustrator пожалуй одна из главных конкурентов Adobe Photoshop. Она уже захватила значительную часть веб-дизайна. Именно этой программе дают предпочтение большинство веб-дизайнеров. В ней много возможностей именно для веб-моделирования. А именно вся векторная графика: иконки, векторные текстуры, логотипы, векторные иллюстрации создаются именно в нем. А потом при помощи этой же программы экспортируются в масштабируемую веб-графику SVG. Для графического же дизайнера иллюстратор программа номер один. Потому что все логотипы, фирстили, простая печатная продукция, не требующая постраничной верстки (визитки, например) также создаются в иллюстраторе. А еще в нем собираются презентации и гайдлайны для клиентов.

3. Coggle.it

Coggle.it это одна из луших программ для майндмаппинга. Для этой задачи многим может понравиться mindmeister, но после создания одной карты, она требует подписку, и многим, конечно же могут не подойти платные программы. Поэтому Coggle.it это отличный бесплатный аналог для майндмаппинга. Coggle не ограничивает в количестве карт, их можно расшаривать другим плзователям, сохранять в картинках и выглядят они достаточно симпатично и удобно.

P.S. mind map - это очень удобный и универсальный инструмент для разработки концепций, проектирования карты сайта, пользовательских сценариев. Как раз с него обычно начинается веб-дизайн.

4. InvisionApp

Очень удобный инструмент для прототипирования. С его помощью Можно не только сразу проработать всю структуру сайта и расставить ссылку, но и расставить такие приятные мелочи как sticky меню или сделать глобальную навигацию. Минус в том, что Invision снова платный и предлагает только один бесплатный прототип. Если вы не ведете по 5 проектов одновременно, этого бывает достаточно.

Есть совершенно бесплатная альтернатива Marvel App, но его функционал намного проще и работать с прототипами все-таки сложнее.


5. Evernote

можно использовать бсплатный вариант, ее возможностей вполне хватает. В Evernote очень удобно хранить брифы, заметки, ссылки, вести органайзер. Это записная книжка по сути, синхронизируется с несколькими вашими устройствами, можно делиться заметками или блокнотами. Многие в нем работают, поэтому очень удобно и для командной коммуникации тоже.

6. Toggl
Тайм треккер для внутреннего использования. Он не такой продвинутый как TimeDoctor, который снимает скриншоты с экрана и отправляет их клиенту каждый 3 минуты. Но toggl помогает правильно распределить время, сосредоточиться и посчитать сколько же времени вы тратите на ту или иную работу. Поработайте с ним годик и вы будете легко оценивать по часам любой проект и соответственно будете называть стоимость не наугад, а понимая сколько на самом деле ваших усилий для конкретного проекта понадобиться.

7. Adobe Lightroom
Редко, но все же использую для быстрой обработки фотографий. Фотошоп, конечно, обладает всеми возможностями лайтрума, но если у вас есть пакет из сотни фотографий для проекта и все они сыроваты, вы можете их быстро прогнать через лайтрум и получить очень приятные фото для эскизного макета, к примеру.

Дополнительный программы.

1. Pinterest

это онлайн сервис где можно смотреть разные работы и почерпать вдохновление.
Ищите примеры работ и составляете из них доску настроения, включающую шрифты, цвета, стиль, которые помогут вам для вдохновения в конкретном проекте.

2. Adobe After Effect
можно использовать для анимации. К примеру, анимации иконок, видеороликов, презентации анимационных сайтов.

3. Cinema 4D

очень полезная вещь для дизайнера: с этой программой можно очень быстро и легко создавать потрясающе красивые вещи для чего угодно: будь это веб или фирстиль.

\section{psychologické metódy}

Визуальная иерархия сайта\newline
\newline

Для того, чтобы улучшить интерфейс страниц веб-сайта, полагаться только на интуицию не получится.Надо следовать набору строгих правил. Правил визуальной иерархии. Это одна из составляющих, которая во многом определяет эффективность того или иного сайта. Иерархия — один из главных ключей к гармоничному изображению. Любая композиция состоит из элементов, и эти элементы должны быть правильно сбалансированы в пространстве, и неважно плоское оно или трехмерное. Визуальная иерархия на сайте — это организация и оформление информации таким образом, чтобы посетитель мог быстро разобраться с интерфейсом и отличить главное от второстепенного. Визуальная иерархия призвана упорядочить информацию на странице сайте или приложения.  Люк Вроблевски, директор по продукту в Google, в статье
\begin{comment}
(http://static.lukew.com/pageheirarchy_lukew_03192008.pdf)
\end{comment}
«Коммуникация через визуальную иерархию» говорит, что в итоге визуальная подача любой веб-страницы должна помогать посетителю быстро найти ответы на три вопроса: Что это такое?(Польза), как я могу это использовать?(Удобство испльзования), Зачем мне это нужно?(Необходимость). За интуитивной конструкцией стоит не хитрое оформление, а прежде всего тщательный анализ и структурирование контента. Нам легче усваивать информацию, упорядоченную при соблюдении некоторой визуальной структуры. Другими словами, блоки сплошного текста имеют небольшую коммуникативную ценность.
\newline
6 основных правил визуальной иерархии
1. РАЗМЕР ЭЛЕМЕНТОВ ИМЕЕТ ЗНАЧЕНИЕ
Очевидный, но от того не менее действенный способ сделать один элемент более весомым, чем другой — сделать его больше.
Чем внушительнее элемент на странице, тем больше он внимания привлечет. Это естественно: наш взгляд останавливается в первую очередь на более внушительных объектах. Вот пример: (photo)
Ваш взгляд автоматически остановился на самом крупном элементе. Для того, чтоб обратить внимание на более мелкие элементы, требуется дополнительное усилие. 
Как вывод делаем? Размер и важность элементов на Вашей странице коррелируются. Страница должна содержать элементы разных размеров, иначе Вы рискуете запутать посетителей

2. НУЖНО ДОБАВИТЬ ЦВЕТА
С помощью цветов можно выделить одни элементы, скрыть другие. Выделенные цветом элементы бросаются в глаза сразу; используя цвет, можно сбалансировать расположение элементов на странице, выделив те, которые должны притягивать к себе внимание. 
Поэтому при создании сайта с цветом нужно обращаться очень аккуратно. В большинстве случаев два цвета на сайте вполне достаточно — один основной, второй дополнительный, для акцентов. В этом случае вам будет проще организовать визуальную иерархию. А вот и пример: (photo)
Здесь все элементы идентичные по форме, но цвет выделяет один из них. Этот цвет и привлекает Ваш взгляд в первую очередь. Также можно ориентировать посетителей, фокусируя внимание с помощью контрастов.

3. Повторение и группировка

Придать значимость определенному объекту можно не только сделав его больше, но и используя прием повторения. Много небольших элементов, расположенных рядом, могут иметь не меньший вес, чем один крупный элемент. Частая ошибка в оформлении контента на сайте — непродуманные расстояния между разными смысловыми блоками. Человек не может быстро, интуитивно понять что к чему относится и из-за этого плохо воспринимает информацию. Например, на странице расположена секция «О продукте» и следом «Преимущества». Из-за того, что заголовок второй секции находится слишком близко к первой, возникает путаница. Это не критично, но раздражает человека, так как заставляет напрягаться. Исправить это просто — сделать расстояния вокруг секции больше, чем между элементами внутри нее. (photo)

4. Белое (пустое) пространство

В процессе дизайна важно не переусердствовать. Главная задача — достигать эффекта минимальными средствами. Не надо считать его пустым пространством, оно скорее является дополнительным элементом наполнения страницы. Простой белый фон является таким же участником общей композиции, как и находящиеся на нем элементы. И вот пример страницы, которую мы все знаем: Google. (photo) 
Страница не выглядит очень уж пустой, при этом отрицательное пространство позволяет посетителю сконцентрироваться на главном - поисковой строке. 

5. ОПРЕДЕЛИТЕ ШРИФТ ВАШЕГО ТЕКСТА

Выбор шрифта Ваших текстов - это тоже очень важная составляющая визуальной иерархии, которая работает. Даже если Вы решите использовать единый шрифт на всём сайте, Вы должны составить типографическую иерархию, которая включает все возможные вариации шрифтов на Вашем сайте. Базовое правило графического дизайна требует, чтобы мы не использовали больше 3 шрифтов, чтобы не создавать непоследовательный, разобранный дизайн:

Первый уровень: самый внушительный, привлекающий внимание посетителя.
здесь располагается самый приоритетный контент; заголовки на этом уровне бросаются в глаза и считываются прежде остальных.

Второй уровень: элементы, которые дополняют первичные заголовки.
на этом уровне находятся подзаголовки, названия вкладок — все то, что обозначает главные «темы». Именно на этом уровне располагаются элементы, по которым пользователь попадает в разделы сайта.

Третий уровень: основной текст. к этому уровню можно отнести текстовые блоки, описания изображений, коротие заметки — все то, что раскрывает тему, но не служит ее названием. Именно здесь вы можете подробно рассказать о всех деталях вашего проекта.

6. Композиция на странице


Когда человек впервые видит изображение — будь то картина, веб-страница или журнал, — он неосознанно следует одной из схем анализа увиденного. У каждого, конечно, свои привычки, но можно определить парочку тенденций. Самые главные это схема F или схема Z. То есть прежде чем приступить к внимательному изучению любого изображения, зритель как бы «сканирует» его, перемещая взгляд по одной из этих воображаемых букв. Это стоит держать в голове, выстраивая элементы на странице. 

Схема Z : Это правило визуальной иерархии применимо в случае, если страница содержит мало текста и много графических элементов. Посетитель Х рефлекторно смотрит по схеме Z: часто из-за наличия горизонтального меню. Композицию таких страниц можно условно разделить на три блока: верхний с названием и заголовками, центральный с изображением и нижний с дополнительной информацией и призывом к действию.

Вот несколько полезных советов: 
1. Отделите, если возможно, цвет фона сайта от контента сайта, чтобы кадрировать визуальный путь посетителя. 
2. Поставьте логотип в верхний левый угол страницы. Это первое, что посетитель видит. 
3. Расположите призывы в диагонали Z и убедитесь, что кнопка CTA находится в левой части страницы, следуя визуальному пути посетителей. 

Схема F: как правило, работает на страницах с большим количеством текста, где основной текстовой блок может быть расположен в широкой вертикальной колонне слева, в то время как справа находятся заголовки других статей, которые глаз выхватывает точечно. Схема в форме F - это скорее наблюдение, чем правило дизайна. Исследование Nielsen Norman Group которое основно на наблюдении за 232 пользователями показало, что посетитель Х прочитает первые строки текста полностью. Затем он пролистывает и читает всё меньше слов в строке, а потом и вовсе бросает. Что позволяет получить путь в форме F. (photo)
Если посетители сайта читают контент по такому сценарию, то они, скорее всего, пропускают суть контента. Это может значить, что им не очень интересно, или структура неудачная. Из этого следует что:
- Пользователи редко будут читать каждое слово вашего текста.
- Первые два абзаца являются самыми важными и должны содержать что-то, что зацепит посетителя.
- Начинайте абзацы, подзаголовки и списки с ключевых слов, которые привлекают внимание.
Если вдруг схема похожа на Е, то это означает, что они не теряют интерес во время прочтения. 


\newline
- что такое визуальная иерархия \newline
- создание иерархии, методы \newline
- прием "перевернутой пирамиды" \newline
- приемы создания иерархии сайта \newline
- схема Z и F \newline
- правильное повторение и группировка \newline
- пустое пространство \newline
- как проверить визуальную иерархию \newline


\end{document}
