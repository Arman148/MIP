% Metódy inžinierskej práce

\documentclass[10pt,twoside,slovak,a4paper]{article}

\usepackage[slovak]{babel}
%\usepackage[T1]{fontenc}
\usepackage[IL2]{fontenc} % lepšia sadzba písmena Ľ než v T1
\usepackage[utf8]{inputenc}
\usepackage{graphicx}
\usepackage{url} % príkaz \url na formátovanie URL
\usepackage{hyperref} % odkazy v texte budú aktívne (pri niektorých triedach dokumentov spôsobuje posun textu)

\usepackage{cite}
%\usepackage{times}

\pagestyle{headings}

\title{Web modelovanie: prečo je to potrebné\thanks{Semestrálny projekt v predmete Metódy inžinierskej práce, ak. rok 2021/22, vedenie: Arman Vardanyan}} % meno a priezvisko vyučujúceho na cvičeniach

\author{Arman Vardanyan\\[2pt]
	{\small Slovenská technická univerzita v Bratislave}\\
	{\small Fakulta informatiky a informačných technológií}\\
	{\small \texttt{xvardanyan@stuba.sk}}
	}

\date{\small 09.10.2021} % upravte

\begin{document}

\maketitle

\begin{center}
    
\end{center}

\section{Úvod}

Modelovanie je jedna z hlavných fáz vytvárania webovej stránky. Počas neho odpovedáme na otázky podľa typu: “prečo vytvaráme tu webovu stránku”, “ aké sú naše ciele ”, “ ako implementovať naše nápady ”, a nakoniec “ ako bude náš web vyzerať a fungovať ”. V tomto článku sa pozrieme na to, ako projektovať webovú stránku, hlavné fázy návrhu webových stránok, aké nástroje a programy sa používajú počas práce, prečo nevhodne ignorovať modelovanie web stranky, aj keď projekt je veľmi malý. Prečo minuv čas na projektovanie my ušetrime viacej času a peniaze na nasledujúcich fázach projekta, a aké psychologické metódy používajú profesionáli aby zaujať pozornosť zakaznika. A nakoniec spoločne pochopíme, prečo web modelovanie je potrebne.


\end{document}
